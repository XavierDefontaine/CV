% Copyright 2013 Christophe-Marie Duquesne <chmd@chmd.fr>
% Copyright 2014 Mark Szepieniec <http://github.com/mszep>
% 
% ConText style for making a resume with pandoc. Inspired by moderncv.
% 
% This CSS document is delivered to you under the CC BY-SA 3.0 License.
% https://creativecommons.org/licenses/by-sa/3.0/deed.en_US

\startmode[*mkii]
  \enableregime[utf-8]  
  \setupcolors[state=start]
\stopmode

\setupcolor[hex]
\definecolor[titlegrey][h=757575]
\definecolor[sectioncolor][h=397249]
\definecolor[rulecolor][h=9cb770]

% Enable hyperlinks
\setupinteraction[state=start, color=sectioncolor]

\setuppapersize [A4][A4]
\setuplayout    [width=middle, height=middle,
                 backspace=20mm, cutspace=0mm,
                 topspace=10mm, bottomspace=20mm,
                 header=0mm, footer=0mm]

%\setuppagenumbering[location={footer,center}]

\setupbodyfont[11pt, helvetica]

\setupwhitespace[medium]

\setupblackrules[width=31mm, color=rulecolor]

\setuphead[chapter]      [style=\tfd]
\setuphead[section]      [style=\tfd\bf, color=titlegrey, align=middle]
\setuphead[subsection]   [style=\tfb\bf, color=sectioncolor, align=right,
                          before={\leavevmode\blackrule\hspace}]
\setuphead[subsubsection][style=\bf]

\setuphead[chapter, section, subsection, subsubsection][number=no]

%\setupdescriptions[width=10mm]

\definedescription
  [description]
  [headstyle=bold, style=normal,
   location=hanging, width=18mm, distance=14mm, margin=0cm]

\setupitemize[autointro, packed]    % prevent orphan list intro
\setupitemize[indentnext=no]

\setupfloat[figure][default={here,nonumber}]
\setupfloat[table][default={here,nonumber}]

\setuptables[textwidth=max, HL=none]
\setupxtable[frame=off,option={stretch,width}]

\setupthinrules[width=15em] % width of horizontal rules

\setupdelimitedtext
  [blockquote]
  [before={\setupalign[middle]},
   indentnext=no,
  ]


\starttext

\section[title={Xavier Defontaine},reference={xavier-defontaine}]

\startblockquote
{\bf Full Stack Junior Developer} looking for a role across Backend
and/or Frontend.
\stopblockquote

\thinrule

\startplacetable[location=none]
\startxtable
\startxtablehead[head]
\startxrow
\startxcell \goto{About}[about-me] \stopxcell
\startxcell \goto{Skills}[skills] \stopxcell
\startxcell \goto{Projects}[projects] \stopxcell
\startxcell \goto{Education}[education] \stopxcell
\startxcell \goto{Experience}[experience] \stopxcell
\stopxrow
\stopxtablehead
\stopxtable
\stopplacetable

\thinrule

\subsection[title={About me},reference={about-me}]

I have recently graduated from Makers Academy, having realised a
lifelong passion for software development. I am passionate about Ruby
and Javascript with a thirst for discovering new languages and
frameworks.

With a background in Logistics & Supply Chain and having spent 7 years
planning worldwide tours for global artist management agencies, I have
developed a strong set of problem solving, collaborative and stakeholder
management skills.

I love programming as it gives me the freedom to create and innovate and
it offers the opportunity to push exciting new boundaries while
improving user experiences.

Throughout my career, I would always think of ways to improve our
processes while keeping abreast of innovations in the sector and I am
now particularly excited to be able to use my technical knowledge to
build solutions to help solve real world problems, whether in my
personal projects or in a workplace.

\subsection[title={Skills},reference={skills}]

\subsubsection[title={Tech Skills 🏋️‍},reference={tech-skills}]

\startplacetable[location=none]
\startxtable
\startxtablehead[head]
\startxrow
\startxcell Languages \stopxcell
\startxcell Technologies \stopxcell
\startxcell Testing \stopxcell
\startxcell Concepts \stopxcell
\startxcell Tools \stopxcell
\stopxrow
\stopxtablehead
\startxtablebody[body]
\startxrow
\startxcell JavaScript \stopxcell
\startxcell React \stopxcell
\startxcell Jest \stopxcell
\startxcell XP/Agile \stopxcell
\startxcell VSCode \stopxcell
\stopxrow
\startxrow
\startxcell NodeJS \stopxcell
\startxcell Express \stopxcell
\startxcell Jasmine \stopxcell
\startxcell TDD/BDD \stopxcell
\startxcell GCP/AWS \stopxcell
\stopxrow
\startxrow
\startxcell Ruby \stopxcell
\startxcell Sequelize \stopxcell
\startxcell Capybara \stopxcell
\startxcell OO Design \stopxcell
\startxcell Git \stopxcell
\stopxrow
\startxrow
\startxcell SQL \stopxcell
\startxcell Ruby on Rails \stopxcell
\startxcell RSpec \stopxcell
\startxcell Remote working \stopxcell
\startxcell TablePlus \stopxcell
\stopxrow
\startxrow
\startxcell HTML5 \stopxcell
\startxcell Sinatra \stopxcell
\startxcell Selenium \stopxcell
\startxcell Pair programming \stopxcell
\startxcell OSX \stopxcell
\stopxrow
\startxrow
\startxcell CSS3 \stopxcell
\startxcell jQuery \stopxcell
\startxcell  \stopxcell
\startxcell CI/CD \stopxcell
\startxcell Bootstrap \stopxcell
\stopxrow
\startxrow
\startxcell Markdown \stopxcell
\startxcell PostgreSQL \stopxcell
\startxcell  \stopxcell
\startxcell Git workflow \stopxcell
\startxcell  \stopxcell
\stopxrow
\startxrow
\startxcell  \stopxcell
\startxcell TravisCI \stopxcell
\startxcell  \stopxcell
\startxcell RESTful APIs \stopxcell
\startxcell  \stopxcell
\stopxrow
\stopxtablebody
\startxtablefoot[foot]
\startxrow
\startxcell  \stopxcell
\startxcell Heroku \stopxcell
\startxcell  \stopxcell
\startxcell MVC Pattern \stopxcell
\startxcell  \stopxcell
\stopxrow
\stopxtablefoot
\stopxtable
\stopplacetable

\subsubsection[title={Soft Skills},reference={soft-skills}]

\subsubsubsection[title={Problem solving
🕵🏻‍♂️},reference={problem-solving}]

I always approach solving problems by breaking them down into manageable
chunks. I am {\bf patient} and never deterred by a problem.

A good example of this was deploying a finished product in just over a
week using entirely new technologies to us for my final project at
Makers - we came up against multiple challenges which we overcame by
mapping how each actor interacted, outlining the flow in the code,
narrowing down the issue, then each reading up on them and sharing what
we learned to inform our next sprint.

While working at MN2S, I led a team of 5 logistics coordinators in a
highly demanding environment, where we were constantly faced with
complex issues relating to artist tours. I taught the team to remain
{\bf calm} and worked with them to find the {\bf cleanest solution}.

\subsubsubsection[title={Stakeholder management
🤝},reference={stakeholder-management}]

Whilst working in artist management, I successfully managed high stakes
relationships with a variety of key stakeholders, including press and
legal teams, event promoters, artist managers and the artists
themselves, to ensure the success of worldwide tours and individual
shows.

This resulted in the band Boney M trusting me to tour manage them in
Thailand and DJ Jazzy Jeff's team agreeing to
\useURL[url1][https://www.bbc.co.uk/news/uk-england-lancashire-41051831][][reunite]\from[url1]
with Will Smith for a one-off show in Blackpool in 2017 which they
hadn't done in decades.

\subsubsubsection[title={Adaptable ☯︎},reference={adaptable}]

I always try to operate using a {\bf growth mindset}, by seeking out
criticism, changing my approach if I meet a difficult challenge and
gathering different perspectives.

This served me well when working on my final group project at Makers as
it meant I could quickly learn new tools and motivate the team when we
were struggling. I pride myself on being {\bf empathic}, a good listener
and have the emotional intelligence to navigate through difficult
situations. This meant I could play an important role in keeping the
project team on track through {\bf positive leadership}.

While working in music, I often had to adapt my communication style to
build good relationships and explain complex concepts (e.g.~technical,
legal etc) to different audiences.

\subsection[title={Projects 👨🏻‍💻},reference={projects}]

\subsubsection[title={\useURL[url2][https://github.com/XavierDefontaine/ProcessReviewSite][][Process
Reviews Dashboard]\from[url2]},reference={process-reviews-dashboard}]

{\em \quotation{A web app for reviewers and students to book reviews and
track their progress visually over time.}}

In this final group project at Makers, I:

\startitemize[packed]
\item
  Challenged myself by learning an entirely new stack which was not
  taught on the course and deployed in just over a week.
\item
  Wrote the frontend with {\bf Javascript/React} and {\bf Bootstrap} for
  extra styling
\item
  Wrote the backend in {\bf NodeJS/Express}
\item
  Hosted the database on {\bf AWS} via {\bf Heroku} and managed via
  {\bf PostgreSQL/Sequelize}
\item
  Used {\bf Bcrypt/Passport} for authentication,
\item
  Deployed via {\bf Heroku} with continuous integration via Travis-CI
\stopitemize

I particularly loved to:

\startitemize[packed]
\item
  Work as part of a team and take ownership of features across backend
  and frontend
\item
  Gain the confidence to take a project from scoping requirements
  through to actual launch
\item
  Learn how to deploy a web app using the server to serve React's build
\item
  Showcase our project to a wider audience
  (\useURL[url3][https://www.youtube.com/watch?v=iJvfMVrU9Vk][][presentation
  on YouTube]\from[url3]).
\stopitemize

Nice to have, if we had more time:

\startitemize[packed]
\item
  Making the frontend more responsive
\item
  Fixing graphs rendering issues when passing props down from the
  Profile
\item
  Implementing Hash History as to avoid 404s on refresh or gracefully
  handle them with a catch-all.
\stopitemize

\subsubsection[title={\useURL[url4][https://github.com/XavierDefontaine/acebook-robotlizard][][Acebook
Robot-lizard]\from[url4]},reference={acebook-robot-lizard}]

{\em \quotation{A web app inspired by Facebook for lizards and robots
who like to hack.}}

In this group project, we used:

\startitemize[packed]
\item
  {\bf Ruby on Rails} for the backend
\item
  {\bf HTML/CSS} for the frontend
\item
  {\bf Devise} for Authentication and emails management
\item
  {\bf Capybara/RSpec} for TDD
\item
  Database management with {\bf PostgreSQL} and {\bf Google Cloud }for
  media files
\item
  Deployed via {\bf Heroku}.
\stopitemize

\subsubsection[title={\useURL[url5][https://github.com/XavierDefontaine/Bank-Tech-Test][][Bank
Tech Test]\from[url5]},reference={bank-tech-test}]

{\em \quotation{A solo project and small test-driven program to interact
with the command line and log finances.}}

Here, I worked off acceptance criterias to create user stories and write
a program in Ruby / RSpec using BDD / best OOP practices.

\subsubsection[title={\useURL[url6][https://github.com/XavierDefontaine/Gilded-Rose-tech-test][][Gilded
Rose]\from[url6]},reference={gilded-rose}]

{\em \quotation{A solo project to work with and refactor legacy code.}}

On this challenge, I had to add a new product to a legacy codebase that
has gone
\useURL[url7][https://github.com/emilybache/GildedRose-Refactoring-Kata/blob/master/ruby/gilded_rose.rb][][out
of hand]\from[url7]. I started by writing
\useURL[url8][https://github.com/XavierDefontaine/Gilded-Rose-tech-test/blob/master/spec/gilded_rose_spec.rb][][tests
with 100\letterpercent{} coverage]\from[url8] via RSpec and encapsulated
behaviour to be able to
\useURL[url9][https://github.com/XavierDefontaine/Gilded-Rose-tech-test/blob/master/lib/refactored_gilded_rose.rb][][refactor
and add the new product]\from[url9]. If I had more time then, I would
have looked at moving each item into respective classes using SRP and
dependency injection.

\subsection[title={Education 🎓},reference={education}]

\subsubsubsection[title={Makers Academy (Sep 2020 - Nov
2020)},reference={makers-academy-sep-2020---nov-2020}]

A highly selective and 12 week intensive {\bf software development
bootcamp}.

\startitemize[packed]
\item
  Built full-stack applications in Ruby, Rails, JavaScript, SQL
  (PostgreSQL, ActiveRecord) and HTML/CSS.
\item
  Extensively practiced TDD principles using RSpec, Jasmine and Capybara
\item
  Learnt and applied architectural patterns including MVC and DDD
\item
  Pair programmed extensively during design and development of
  applications
\item
  Provided code reviews focusing on OOP, DRY, SRP and RESTful API design
\item
  Deployed to, and interacted with, cloud platforms including AWS, GCP
  and Heroku
\item
  Integrated CI/CD workflows using Travis-CI and Github
\item
  Development driven by Agile project management principles using Github
  Projects and UML diagrams
\stopitemize

\subsubsubsection[title={Marseille University - France (2009 -
2011)},reference={marseille-university---france-2009---2011}]

\startitemize[packed]
\item
  Bachelor's degree in Logistics and Supply Chain Management
\stopitemize

\subsubsubsection[title={University of Nice - France (2007 -
2009)},reference={university-of-nice---france-2007---2009}]

\startitemize[packed]
\item
  HND in Quality, Logistics and Organisation
\stopitemize

\subsubsubsection[title={Lycée Albert 1er - Monaco
(2007)},reference={lycée-albert-1er---monaco-2007}]

\startitemize[packed]
\item
  A-level in Electrotechnics
\stopitemize

\subsubsubsection[title={Music Academy of Monaco (2001 -
2007)},reference={music-academy-of-monaco-2001---2007}]

\startitemize[packed]
\item
  Higher Diploma in Music Theory
\stopitemize

\subsection[title={Experience ☕️},reference={experience}]

{\bf Paradigm Talent Agency} - London (Jan 2019 - Dec 2020)\crlf
{\em Executive Assistant to 2 Senior Agents}

{\bf MN2S Booking Agency} - London (Aug 2014 - Jan 2019)\crlf
{\em Logistics Coordinator & Admin Manager}

\startitemize[packed]
\item
  Led a team of 5 assistants overseeing a roster of 200+ artists across
  20 agents and its logistics.
\item
  Set up a corporate travel unit and provided expertise to boost our
  margins by up to 20\letterpercent{}.
\stopitemize

{\bf The Big Noise Festival} - Elephant & Castle, London (2014)\crlf
{\em Booker/Production assistant}

\startitemize[packed]
\item
  Facilitated the procurement of sound equipment, live streamed and
  booked artists.
\stopitemize

{\bf The Trailer TV} - Deptford, London (2013 - 2015)\crlf
{\em Co-founder - \quotation{Fortnightly event showcasing emerging
artists through its live broadcast of performances, interviews and music
videos in a lorry trailer.}}

\startitemize[packed]
\item
  Designed tech workshops for local arts charities to teach kids on how
  to live stream and DJ.
\stopitemize

{\bf ULYSSE Transport} - Nice, France (2010 - 2011)\crlf
{\em Transport Assistant and Fleet Officer}

\startitemize[packed]
\item
  Coordinated 20 drivers and reduced Director's admin time by creating
  automated tools to monitor the national fleet via Excel (insurance
  expiry, fines due dates, car locations, etc).
\stopitemize

{\bf HELP Transport} - Nice, France (2009 - 2010)\crlf
{\em Transport Assistant Officer}

\startitemize[packed]
\item
  Coordinated over 10 truck drivers internationally and increased our
  chartering margin by 10\letterpercent{} by negotiating contracts and
  optimising internal flows (invoicing, database standardisation etc).
\stopitemize

\subsection[title={Interests 🎉},reference={interests}]

\startitemize[packed]
\item
  {\bf Martial Arts}: I practice wing-chun (when there is no lock down).
\item
  {\bf Music}: from playing classical/baroque music on the recorder as a
  kid to heavy metal as rhythm guitar in high school or obsessively
  listening to electronic music now, music has always been a passion of
  mine -
  \useURL[url10][https://open.spotify.com/playlist/37i9dQZF1EM51tz1HB9yQx][][my
  top 2020 Spotify songs]\from[url10].
\item
  {\bf Restoration}: I love restoring mid-century furniture, gutting
  things around the house to see how they work and {\em sometimes} fix
  them.
\item
  {\bf Challenging}: I am fascinated by concepts totally alien to me
  especially those around the tech community whether philosophical (e.g
  \useURL[url11][https://www.lesswrong.com/][][Lesswrong]\from[url11]'s
  rationalism & AI) or more technical (e.g Hacker News) often leading me
  down rabbits holes for hours on end.
\item
  {\bf Currently Reading}: Metaprogramming Ruby by Paolo Perrota - Full
  Spectrum 3 - Thinking, Fast and Slow by Daniel Kahneman
\item
  {\bf {\em Virtually} Attending}: LRUG (London Ruby User Group) / BBC
  tech meetups.
\stopitemize

\thinrule

\startblockquote
\useURL[url12][mailto:xdefontaine@gmail.com][][xdefontaine@gmail.com]\from[url12]
• \useURL[url13][tel:+44\%207\%20845\%20585\%20137][][+44 7 845 585
137]\from[url13] • Deptford, London, UK
\stopblockquote

\thinrule

{\em Markdown CV made possible with
\useURL[url14][https://pandoc.org/][][Pandoc]\from[url14] and
\useURL[url15][https://github.com/mszep/pandoc_resume][][mszep]\from[url15].}

\stoptext
